\documentclass{article}

\usepackage{siunitx}
\usepackage{hyperref}
\usepackage[english]{datetime2}

\begin{document}

\today

Dear Editors of Nano Letters:

Please find enclosed our manuscript titled “Nanoscale Mapping of Intrinsic Slip Length at Solid-Liquid Interfaces” by Haruya Ishida, Hideaki Teshima, Koji Takahashi, Vishwanath Ganesan, and Nenad Miljkovic, for consideration as a Letter in Nano Letters.

A persistent problem in interfacial physics is that reported slip lengths at solid-liquid interfaces—especially on “hydrophobic” surfaces—span orders of magnitude even for nominally similar systems, blurring a basic question: does hydrophobicity actually produce slippage? Our work resolves this long-standing inconsistency by providing the missing experimental capability: we directly visualize the slip boundary condition at the nanoscale while simultaneously imaging the surface geometry that can masquerade as slip.

Specifically, we develop a frequency-modulation AFM approach that boosts hydrodynamic (slip) sensitivity by $159\times$ and enables co-registered nanoscale mapping of slip length and surface topography. This simultaneous mapping is crucial: it allows us to pinpoint and separate intrinsic (“true”) slip on genuinely flat regions from “apparent slip” generated by unavoidable nanoscale heterogeneities. In one measurement, we can see where slip is real, where it is an artifact, and why.

Our key findings are summarized as follows:
\begin{enumerate}
\item Most "hydrophobic slip" is not intrinsic slip. Using nanoscopically flat substrates spanning intrinsic contact angles from \SI{3}{\degree} to \SI{111}{\degree}, we find that the true slip length on them is essentially zero (sub-nanometer; even Teflon: $0.8 \pm  \SI{0.8}{\nano\metre}$). Our dataset quantitatively follows the molecular-dynamics-predicted wettability scaling, thereby reconciling a decades-old mismatch between simulations and experiments.

\item Apparent slip can be directly identified—and removed. Our mapping shows that nanoscale geometry and nanobubbles generate pronounced local changes in the apparent slips. This is the key mechanism behind the notorious irreproducibility in the literature: the “slip length” many techniques report is often a convolution of boundary condition and nanoscale geometry.

\item Graphite is a genuine exception, but only under specific conditions. On graphite in deionized water, we measure a large slip length of $43.2 \pm \SI{5.8}{\nano\metre}$—yet this slippage collapses to a no-slip condition in electrolyte solutions. This striking on/off behavior highlights the decisive influence of interfacial charge/ion adsorption in pinning water at otherwise ultra-smooth, chemically homogeneous carbon interfaces.
\end{enumerate}

More broadly, our co-registered FM-AFM approach turns the slip boundary condition from a fitted parameter into a directly imaged, quantitative field variable. This platform enables systematic tests of how ions/charge, soft or porous interfaces, and nanoscale gas phases regulate interfacial transport, providing reproducible benchmarks for theory and guiding predictive nanofluidic design. We believe this manuscript will be of strong interest to the readership of Nano Letters, given its emphasis on nanoscale physics of fluids, surfaces and interfaces, and transport phenomena.

We confirm that this manuscript has not been published elsewhere and is not under consideration in whole or in part by another journal. The authors declare no competing interests.

We would like to recommend the following potential referees: 
\begin{itemize}
\item Prof. Lydéric Bocquet\\
(École Normale Supérieure, France; lyderic.bocquet@ens.fr) 
\item Prof. Timothée Mouterde\\
(The University of Tokyo, Japan; mouterde@g.ecc.u-tokyo.ac.jp) 
\item Prof. Laurent Joly\\
(Université Lyon 1, France; laurent.joly@univ-lyon1.fr)
\item Prof. Enrique Wagemann\\
(Universidad de Concepción, Chile; ewagemann@udec.cl)
\item Prof. Sushanta Mitra\\
(University of Waterloo, Canada; skmitra@uwaterloo.ca)
\item Prof. Daniel Orejon\\
(The University of Edinburgh, UK; D.Orejon@ed.ac.uk)
\item Prof. Daniel J. Preston\\
(Rice University, USA, djp@rice.edu)
\item Prof. Shalabh Maroo\\
(Syracuse University, USA, scmaroo@syr.edu)
\item Prof. Dion Antao\\
(Texas A\&M University, USA, dantao@tamu.edu)
\end{itemize}

Yours sincerely,\\
Hideaki Teshima (Corresponding author)\\
Associate Professor\\
Department of Aeronautics and Astronautics\\
Kyushu University\\
Nishi-Ku, Motooka 744 Fukuoka 819-0395, Japan\\
Tel: +81-92-802-3050\\
E-mail: hteshima05@aero.kyushu-u.ac.jp

\end{document}