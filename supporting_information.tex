%%%%%%%%%%%%%%%%%%%%%%%%%%%%%%%%%%%%%%%%%%%%%%%%%%%%%%%%%%%%%%%%%%%%%
%% This is a (brief) model paper using the achemso class
%% The document class accepts keyval options, which should include
%% the target journal and optionally the manuscript type. 
%%%%%%%%%%%%%%%%%%%%%%%%%%%%%%%%%%%%%%%%%%%%%%%%%%%%%%%%%%%%%%%%%%%%%
\documentclass[journal=jacsat,manuscript=article]{achemso}

%%%%%%%%%%%%%%%%%%%%%%%%%%%%%%%%%%%%%%%%%%%%%%%%%%%%%%%%%%%%%%%%%%%%%
%% Place any additional packages needed here.  Only include packages
%% which are essential, to avoid problems later. Do NOT use any
%% packages which require e-TeX (for example etoolbox): the e-TeX
%% extensions are not currently available on the ACS conversion
%% servers.
%%%%%%%%%%%%%%%%%%%%%%%%%%%%%%%%%%%%%%%%%%%%%%%%%%%%%%%%%%%%%%%%%%%%%
\usepackage[version=3]{mhchem} % Formula subscripts using \ce{}
\usepackage{siunitx}
\usepackage{graphicx}
\usepackage{xr} % For cross-referencing equations from main.tex
\externaldocument{main}

%%%%%%%%%%%%%%%%%%%%%%%%%%%%%%%%%%%%%%%%%%%%%%%%%%%%%%%%%%%%%%%%%%%%%
%% If issues arise when submitting your manuscript, you may want to
%% un-comment the next line.  This provides information on the
%% version of every file you have used.
%%%%%%%%%%%%%%%%%%%%%%%%%%%%%%%%%%%%%%%%%%%%%%%%%%%%%%%%%%%%%%%%%%%%%
%%\listfiles

%%%%%%%%%%%%%%%%%%%%%%%%%%%%%%%%%%%%%%%%%%%%%%%%%%%%%%%%%%%%%%%%%%%%%
%% Place any additional macros here.  Please use \newcommand* where
%% possible, and avoid layout-changing macros (which are not used
%% when typesetting).
%%%%%%%%%%%%%%%%%%%%%%%%%%%%%%%%%%%%%%%%%%%%%%%%%%%%%%%%%%%%%%%%%%%%%
\newcommand*\mycommand[1]{\texttt{\emph{#1}}}
\setcounter{figure}{0}
\renewcommand{\thefigure}{S\arabic{figure}}
\setcounter{table}{0}
\renewcommand{\thetable}{S\arabic{table}}

%%%%%%%%%%%%%%%%%%%%%%%%%%%%%%%%%%%%%%%%%%%%%%%%%%%%%%%%%%%%%%%%%%%%%
%% Meta-data block
%% ---------------
%% Each author should be given as a separate \author command.
%%
%% Corresponding authors should have an e-mail given after the author
%% name as an \email command. Phone and fax numbers can be given
%% using \phone and \fax, respectively; this information is optional.
%%
%% The affiliation of authors is given after the authors; each
%% \affiliation command applies to all preceding authors not already
%% assigned an affiliation.
%%
%% The affiliation takes an option argument for the short name.  This
%% will typically be something like "University of Somewhere".
%%
%% The \altaffiliation macro should be used for new address, etc.
%% On the other hand, \alsoaffiliation is used on a per author basis
%% when authors are associated with multiple institutions.
%%%%%%%%%%%%%%%%%%%%%%%%%%%%%%%%%%%%%%%%%%%%%%%%%%%%%%%%%%%%%%%%%%%%%
\author{Haruya Ishida}
\affiliation[Kyushu University]
{Department of Aeronautics and Astronautics, Kyushu University, Nishi-Ku, Motooka 744, Fukuoka 819-0395, Japan}
\author{Hideaki Teshima}
\affiliation[Kyushu University]
{Department of Aeronautics and Astronautics, Kyushu University, Nishi-Ku, Motooka 744, Fukuoka 819-0395, Japan}
\email{hteshima05@aero.kyushu-u.ac.jp}
\alsoaffiliation[I2CNER]
{International Institute for Carbon-Neutral Energy Research (WPI-I2CNER), Kyushu University, Nishi-Ku, Motooka 744, Fukuoka 819-0395, Japan}
\author{Koji Takahashi}
\affiliation[Kyushu University]
{Department of Aeronautics and Astronautics, Kyushu University, Nishi-Ku, Motooka 744, Fukuoka 819-0395, Japan}
\alsoaffiliation[I2CNER]
{International Institute for Carbon-Neutral Energy Research (WPI-I2CNER), Kyushu University, Nishi-Ku, Motooka 744, Fukuoka 819-0395, Japan}
\author{Vishwanath Ganesan}
\affiliation[UIUC MechSE]
{Department of Mechanical Science and Engineering, University of Illinois at Urbana-Champaign, Urbana, IL, USA}
\author{Nenad Miljkovic}
\affiliation[I2CNER]
{International Institute for Carbon-Neutral Energy Research (WPI-I2CNER), Kyushu University, Nishi-Ku, Motooka 744, Fukuoka 819-0395, Japan}
\alsoaffiliation[UIUC MechSE]
{Department of Mechanical Science and Engineering, University of Illinois at Urbana-Champaign, Urbana, IL, USA}
\alsoaffiliation[UIUC MRL]
{Materials Research Laboratory, University of Illinois at Urbana-Champaign, Urbana, IL, USA}
\alsoaffiliation[UIUC ECE]
{Department of Electrical and Computer Engineering, University of Illinois at Urbana-Champaign, Urbana, IL, USA}
\alsoaffiliation[iSEE]
{Institute for Sustainability, Energy and Environment (iSEE), University of Illinois at Urbana-Champaign, Urbana, IL, USA}
\alsoaffiliation[ARC]
{Air Conditioning and Refrigeration Center, University of Illinois at Urbana-Champaign, Urbana, IL, USA}

%%%%%%%%%%%%%%%%%%%%%%%%%%%%%%%%%%%%%%%%%%%%%%%%%%%%%%%%%%%%%%%%%%%%%
%% The document title should be given as usual. Some journals require
%% a running title from the author: this should be supplied as an
%% optional argument to \title.
%%%%%%%%%%%%%%%%%%%%%%%%%%%%%%%%%%%%%%%%%%%%%%%%%%%%%%%%%%%%%%%%%%%%%
\title[Title_supporting_information]
  {Supportin Information\\
  Hydrophobic Surfaces Are Not Slippery: Insights from Nanoscale Slip Length Mapping}

%%%%%%%%%%%%%%%%%%%%%%%%%%%%%%%%%%%%%%%%%%%%%%%%%%%%%%%%%%%%%%%%%%%%%
%% Some journals require a list of abbreviations or keywords to be
%% supplied. These should be set up here, and will be printed after
%% the title and author information, if needed.
%%%%%%%%%%%%%%%%%%%%%%%%%%%%%%%%%%%%%%%%%%%%%%%%%%%%%%%%%%%%%%%%%%%%%
\abbreviations{IR,NMR,UV}
\keywords{American Chemical Society, \LaTeX}

%%%%%%%%%%%%%%%%%%%%%%%%%%%%%%%%%%%%%%%%%%%%%%%%%%%%%%%%%%%%%%%%%%%%%
%% The manuscript does not need to include \maketitle, which is
%% executed automatically.
%%%%%%%%%%%%%%%%%%%%%%%%%%%%%%%%%%%%%%%%%%%%%%%%%%%%%%%%%%%%%%%%%%%%%
\begin{document}

%%%%%%%%%%%%%%%%%%%%%%%%%%%%%%%%%%%%%%%%%%%%%%%%%%%%%%%%%%%%%%%%%%%%%
%% The "tocentry" environment can be used to create an entry for the
%% graphical table of contents. It is given here as some journals
%% require that it is printed as part of the abstract page. It will
%% be automatically moved as appropriate.
%%%%%%%%%%%%%%%%%%%%%%%%%%%%%%%%%%%%%%%%%%%%%%%%%%%%%%%%%%%%%%%%%%%%%
\begin{tocentry}

Some journals require a graphical entry for the Table of Contents.
This should be laid out ``print ready'' so that the sizing of the
text is correct.

Inside the \texttt{tocentry} environment, the font used is Helvetica
8\,pt, as required by \emph{Journal of the American Chemical
Society}.

The surrounding frame is 9\,cm by 3.5\,cm, which is the maximum
permitted for  \emph{Journal of the American Chemical Society}
graphical table of content entries. The box will not resize if the
content is too big: instead it will overflow the edge of the box.

This box and the associated title will always be printed on a
separate page at the end of the document.

\end{tocentry}

%%%%%%%%%%%%%%%%%%%%%%%%%%%%%%%%%%%%%%%%%%%%%%%%%%%%%%%%%%%%%%%%%%%%%
%% The abstract environment will automatically gobble the contents
%% if an abstract is not used by the target journal.
%%%%%%%%%%%%%%%%%%%%%%%%%%%%%%%%%%%%%%%%%%%%%%%%%%%%%%%%%%%%%%%%%%%%%
\begin{abstract}
  This is an example document for the \textsf{achemso} document
  class, intended for submissions to the American Chemical Society
  for publication. The class is based on the standard \LaTeXe\
  \textsf{report} file, and does not seek to reproduce the appearance
  of a published paper.

  This is an abstract for the \textsf{achemso} document class
  demonstration document.  An abstract is only allowed for certain
  manuscript types.  The selection of \texttt{journal} and
  \texttt{manuscript} will determine if an abstract is valid.  If
  not, the class will issue an appropriate error.
\end{abstract}

%%%%%%%%%%%%%%%%%%%%%%%%%%%%%%%%%%%%%%%%%%%%%%%%%%%%%%%%%%%%%%%%%%%%%
%% Start the main part of the manuscript here.
%%%%%%%%%%%%%%%%%%%%%%%%%%%%%%%%%%%%%%%%%%%%%%%%%%%%%%%%%%%%%%%%%%%%%
\section{main}

\subsection{Relation between Amplitude Decay and Damping Coefficient}

The total damping coefficient, \(\gamma_{\text{total}}\), acting on the probe can be expressed as the sum of the damping due to the probe sphere, \(\gamma_{\text{tip}}\), and the background damping, \(\gamma_{\text{bulk}}\):

\begin{equation}
  \gamma_{\text{total}} = \gamma_{\text{tip}}(h, b_t, b_s) + \gamma_{\text{bulk}}.
  \tag{S1}
  \label{eqn:s1}
\end{equation}

where \(h\) is probe-substrate distance, and \(b_t\) and \(b_s\) are the slip lengths at the probe and substrate surfaces, respectively. The background damping \(\gamma_{\text{bulk}}\) originates from various sources, such as viscous drag on the cantilever beam and friction due to thermal fluctuations [1]. However, it remains nearly constant and is therefore treated as a constant here. Using the definition of the quality factor \(Q(h) = k/(\omega\gamma_{\text{total}}(h))\), (Eq.~\ref{eqn:s1}) can be rewritten as:

\begin{equation}
  \frac{k}{\omega Q(h)} = \gamma_{\text{tip}}(h, b_t, b_s) + \gamma_{\text{bulk}}.
  \tag{S2}
  \label{eqn:s2}
\end{equation}

At sufficiently large \(h \gg R\), the influence of the substrate becomes negligible and the hydrodynamic drag on the probe reduces to the Stokes resistance, \(\gamma_{\text{tip}} = 6\pi\eta R\). Because this drag on a \SI{300}{\nano\metre} spherical tip is orders of magnitude smaller than the viscous damping acting on the micrometer-scale cantilever, \(\gamma_{\text{tip}}\) becomes negligibly small compared to \(\gamma_{\text{bulk}}\), leading to:

\begin{equation}
  \gamma_{\text{total}} = \gamma_{\text{bulk}} = \frac{k}{\omega Q_{\text{bulk}}}.
  \tag{S3}
  \label{eqn:s3}
\end{equation}

Combining (Eqs.~\ref{eqn:s2} and \ref{eqn:s3}) yields:

\begin{equation}
  \frac{Q_{\text{bulk}}}{Q(h)} = \frac{\gamma_{\text{tip}}(h, b_t, b_s)}{\gamma_{\text{bulk}}} + 1.
  \tag{S4}
  \label{eqn:s4}
\end{equation}

In normal FM-AFM, the oscillation amplitude is stabilized by automatic gain control (AGC). In contrast, in this method, AGC was disabled and the cantilever was driven at a constant power. As a result, the amplitude decay directly reflects the damping, and the following relation between the quality factor and the amplitude holds:

\begin{equation}
  \frac{Q_{\text{bulk}}}{Q(h)} = \frac{A_{\text{bulk}}}{A(h)}.
  \tag{S5}
  \label{eqn:s5}
\end{equation}

Therefore, from (Eqs.~\ref{eqn:s4} and \ref{eqn:s5}), the following relation connecting the ratio of amplitude attenuation to the damping coefficient is obtained:

\begin{equation}
  \gamma_{\text{tip}}(h, b_t, b_s) = \gamma_{\text{bulk}}\left(\frac{A_{\text{bulk}}}{A(h)} - 1\right).
  \tag{S6}
  \label{eqn:s6}
\end{equation}

\subsection{Comparison of Sensitivity between Conventional and Proposed Methods}

We compared the sensitivity of slip length measurements between conventional contact-mode AFM and our newly proposed FM-AFM. With contact-mode AFM, the cantilever deflection \(x\) is measured as a sensor voltage, and the relationship between the deflection sensor voltage \(V_x\) and \(x\) is expressed using the sensitivity \(s\) as \(x = sV_x\). Furthermore, to determine the viscous drag in (Eq.~\ref{eqn:hydrodynamic force}) from the restoring force of the cantilever, \(kx\) (\(k\) is the spring constant), the following relation holds:

\begin{equation}
  F = -ksV_x = -\frac{6\pi\eta R^2 \dot{h}}{h} f^{*}(b_t, b_s).
  \tag{S7}
  \label{eqn:s7}
\end{equation}

In contrast, in the FM-AFM method, the oscillation amplitude \(A\) is recorded as a sensor voltage \(V_A\), with the relationship \(A = sV_A/\sqrt{2}\). From (Eqs.~\ref{eqn:damping coefficient amplitude} and \ref{eqn:damping coefficient slip}) of the manuscript, the following relation is obtained:

\begin{equation}
  \frac{6\pi\eta R^2}{h} f^{*}(b_t, b_s) = \gamma_{\text{bulk}}\left(\frac{V_{A,\text{bulk}}}{V_A(h)} - 1\right).
  \tag{S8}
  \label{eqn:s8}
\end{equation}

Here, the measurement sensitivity is defined as the change in signal voltage with respect to the slip length, \(\partial V/\partial b_s\). Differentiating (Eqs.~\ref{eqn:s7} and \ref{eqn:s8}) yields the sensitivities for each method, \(S_{\text{contact}}\) and \(S_{\text{FM}}\):

\begin{equation}
  S_{\text{contact}} = \frac{\partial V_x}{\partial b_s} = -\frac{6\pi\eta R^2 \dot{h}}{k s h}\frac{\partial f^{*}}{\partial b_s}.
  \tag{S9}
  \label{eqn:s9}
\end{equation}

\begin{equation}
  S_{\text{FM}} = \frac{\partial V_A}{\partial b_s} = -\frac{6\pi\eta R^2 V_{A,\text{bulk}}}{h\gamma_{\text{bulk}}}\left(1 + \frac{6\pi\eta R^2 f^{*}}{h\gamma_{\text{bulk}}}\right)^{-2}\frac{\partial f^{*}}{\partial b_s}.
  \tag{S10}
  \label{eqn:s10}
\end{equation}

From (Eqs.~\ref{eqn:s9} and \ref{eqn:s10}), the ratio of sensitivities is given by:

\begin{equation}
  \frac{S_{\text{FM}}}{S_{\text{contact}}} = \frac{V_{A,\text{bulk}} k s}{\gamma_{\text{bulk}} \dot{h}}\left(1 + \frac{6\pi\eta R^2 f^{*}}{h\gamma_{\text{bulk}}}\right)^{-2}.
  \tag{S11}
  \label{eqn:s11}
\end{equation}

The sensitivity ratio was calculated to be \(S_{\text{FM}}/S_{\text{contact}} = 159\) using typical physical parameters in the measurements: \(V_{A,\text{bulk}} =\) \SI{0.2}{\volt}, \(k =\) \SI{20}{\newton\per\metre}, \(s =\) \SI{10}{\nano\metre\per\volt}, \(\gamma_{\text{bulk}} =\) \SI{2.3}{\micro\newton\second\per\metre}, \(\dot{h} =\) \SI{100}{\micro\metre\per\second}, \(R =\) \SI{300}{\nano\metre}, \(\eta =\) \SI{1}{\milli\pascal\second}, \(f^{*} = 0.63\), and \(h =\) \SI{10}{\nano\metre}. The \(f^{*}\) value corresponds to the slip length of \(b_t =\) \SI{0}{\nano\metre} and \(b_s =\) \SI{10}{\nano\metre}, and the tip substrate distance \(h =\) \SI{10}{\nano\metre} is chosen as a representative value among the range of distances used in the calculation. This indicates that the present method is 159 times more sensitive to slip length than the conventional approach. Consequently, slip length can be measured using much smaller colloidal probes than before.

\subsection{Materials}

\subsubsection{Fabrication of Hydrophilic-Hydrophobic Composite Substrates}

A thermally grown silica surface on a Si wafer was spin-coated with an electron-beam resist (ZEP520A, ZEON Corporation, Japan) at \SI{5000}{\per\minute} to form a \SI{300}{\nano\metre}-thick film. Using an electron-beam lithography system (JSM-6360, JEOL Ltd., Japan / BEAM DRAW, Tokyo Technology Inc., Japan), square patterns of \SI{1}{\micro\metre}\(\times\)\SI{1}{\micro\metre} were drawn at a \SI{2}{\micro\metre} pitch, and the resist in the exposed areas was removed by development. Subsequently, a \SI{5}{\nano\metre}-thick indium tin oxide (ITO) film was deposited by magnetron sputtering (MS-4B-3T, Cosmo System Co., Ltd., Japan). The electron-beam resist was then lifted off by rinsing in dimethylacetamide for \SI{5}{\minute}, leaving ITO patches on the silica substrate. The patterned ITO substrate was treated with \(\mathrm{O}_2\) plasma (PR500, Yamato Scientific Co., Japan) and then immersed in a \SI{1}{\milli\mole\per\litre} ethanol solution of 1H,1H,2H,2H-perfluorooctanephosphonic acid (FOPA) for \SI{1}{\hour} to form a self-assembled monolayer (SAM). Since FOPA SAMs are easily hydrolyzed on silica but remain stable on ITO surfaces, maintaining hydrophobicity for hours in water [2,3], the fabricated composite substrate was immersed in water for \SI{3}{\hour} to selectively remove FOPA from the silica surface.

\subsubsection{Fabrication of Teflon Substrates}

Silica substrates cut to \SI{1}{\centi\metre}\(\times\)\SI{1}{\centi\metre} were ultrasonically cleaned in acetone (US-1KS, SND Co., Ltd., Japan) and treated with \(\mathrm{O}_2\) plasma. After baking at \SI{120}{\degreeCelsius} for \SI{5}{\minute} to remove surface moisture, the substrates were immersed for \SI{1}{\hour} in a \SI{0.05}{\percent} (w/w) toluene solution of FDTS to form an FDTS monolayer. On top of the FDTS-treated substrate, a \SI{0.6}{\percent} (w/w) solution of Teflon AF1600X (DuPont Inc., Wilmington, DE, USA) in FC-770 (3M, St. Paul, MN, USA) was dropped, spin-coated at \SI{4000}{\per\minute} for \SI{1}{\minute}, and baked at \SI{175}{\degreeCelsius} for \SI{10}{\minute}. The FDTS underlayer was used to improve the adhesion of the Teflon coating [4].

\begin{figure}
  \centering
  \includegraphics[width=1.0\linewidth]{figs/mapping_colormaps.png}
  \caption{ (a-d) Topography and (e-h) corresponding slip-length maps of (a, e) mica, (b, f) FDTS, (c, g) Teflon, and (d, h) HOPG. The \(R_a\) values were \SI{0.27}{\nano\metre}, \SI{0.29}{\nano\metre}, \SI{0.64}{\nano\metre}, and \SI{0.35}{\nano\metre} for each surface, respectively. Only the result for HOPG (d, h) was obtained in a \SI{10}{\milli\mole\per\litre} KCl solution and the others were in DI water.}
  \label{figS:mapping_colormaps}
\end{figure}

\begin{table}
  \caption{Contact angle and slip length for various substrates}
  \label{tbl:contact_angle_slip}
  \begin{tabular}{lcc}
    \hline
    Substrate & Contact Angle \(\theta\) (\si{\degree}) & Slip length (\si{\nano\metre}) \\
    \hline
    Mica & \(< \SI{3}{\degree}\) & \(-1.7 \pm \SI{0.3}{\nano\metre}\) \\
    Oxygen plasma-treated silica & \SI{28.1}{\degree} & \(-0.7 \pm \SI{0.6}{\nano\metre}\) \\
    FDTS & \SI{97.9}{\degree} & \(-1.7 \pm \SI{0.2}{\nano\metre}\) \\
    FOPA & \SI{105.0}{\degree} & \(-0.9 \pm \SI{0.6}{\nano\metre}\) \\
    Teflon & \SI{111.1}{\degree} & \(0.8 \pm \SI{0.8}{\nano\metre}\) \\
    HOPG (DI water) & \SI{59.0}{\degree} & \(43.2 \pm \SI{5.8}{\nano\metre}\) \\
    HOPG (\SI{10}{\milli\mole\per\litre} KCl aq.) & \SI{62.8}{\degree} & \(-0.1 \pm \SI{0.7}{\nano\metre}\) \\
    HOPG (\SI{10}{\milli\mole\per\litre} NaCl aq.) & \SI{57.0}{\degree} & \(-2.3 \pm \SI{0.6}{\nano\metre}\) \\
    \hline
  \end{tabular}
\end{table}

\section{Results and discussion}

\subsection{Outline}

The document layout should follow the style of the journal concerned.
Where appropriate, sections and subsections should be added in the
normal way. If the class options are set correctly, warnings will be
given if these should not be present.

\subsection{References}

The class makes various changes to the way that references are
handled.  The class loads \textsf{natbib}, and also the
appropriate bibliography style.  References can be made using
the normal method; the citation should be placed before any
punctuation, as the class will move it if using a superscript
citation style
\cite{Mena2000,Abernethy2003,Friedman-Hill2003,EuropeanCommission2008}.
The use of \textsf{natbib} allows the use of the various citation
commands of that package: \citeauthor{Abernethy2003} have shown
something, in \citeyear{Cotton1999}, or as given by
Ref.~\citenum{Mena2000}.  Long lists of authors will be
automatically truncated in most article formats, but not in
supplementary information or reviews \cite{Pople2003}. If you
encounter problems with the citation macros, please check that
your copy of \textsf{natbib} is up to date. The demonstration
database file \texttt{achemso-demo.bib} shows how to complete
entries correctly. Notice that ``\latin{et al.}'' is auto-formatted
using the \texttt{\textbackslash latin} command.

Multiple citations to be combined into a list can be given as
a single citation.  This uses the \textsf{mciteplus} package
\cite{Johnson1972,*Arduengo1992,*Eisenstein2005,*Arduengo1994}.
Citations other than the first of the list should be indicated
with a star. If the \textsf{mciteplus} package is not installed,
the standard bibliography tools will still work but starred
references will be ignored. Individual references can be referred
to using \texttt{\textbackslash mciteSubRef}:
``ref.~\mciteSubRef{Eisenstein2005}''.

The class also handles notes to be added to the bibliography.  These
should be given in place in the document \bibnote{This is a note.
The text will be moved the the references section.  The title of the
section will change to ``Notes and References''.}.  As with
citations, the text should be placed before punctuation.  A note is
also generated if a citation has an optional note.  This assumes that
the whole work has already been cited: odd numbering will result if
this is not the case \cite[p.~1]{Cotton1999}.

\subsection{Floats}

New float types are automatically set up by the class file.  The
means graphics are included as follows (Scheme~\ref{sch:example}).  As
illustrated, the float is ``here'' if possible.
\begin{scheme}
  Your scheme graphic would go here: \texttt{.eps} format\\
  for \LaTeX\, or \texttt{.pdf} (or \texttt{.png}) for pdf\LaTeX\\
  \textsc{ChemDraw} files are best saved as \texttt{.eps} files:\\
  these can be scaled without loss of quality, and can be\\
  converted to \texttt{.pdf} files easily using \texttt{eps2pdf}.\\
  %\includegraphics{graphic}
  \caption{An example scheme}
  \label{sch:example}
\end{scheme}

\begin{figure}
  As well as the standard float types \texttt{table}\\
  and \texttt{figure}, the class also recognises\\
  \texttt{scheme}, \texttt{chart} and \texttt{graph}.
  \caption{An example figure}
  \label{fgr:example}
\end{figure}

Charts, figures and schemes do not necessarily have to be labelled or
captioned.  However, tables should always have a title. It is
possible to include a number and label for a graphic without any
title, using an empty argument to the \texttt{\textbackslash caption}
macro.

The use of the different floating environments is not required, but
it is intended to make document preparation easier for authors. In
general, you should place your graphics where they make logical
sense; the production process will move them if needed.

\subsection{Math(s)}

The \textsf{achemso} class does not load any particular additional
support for mathematics.  If packages such as \textsf{amsmath} are
required, they should be loaded in the preamble.  However,
the basic \LaTeX\ math(s) input should work correctly without
this.  Some inline material \( y = mx + c \) or $ 1 + 1 = 2 $
followed by some display. \[ A = \pi r^2 \]

It is possible to label equations in the usual way (Eq.~\ref{eqn:example}).
\begin{equation}
  \frac{\mathrm{d}}{\mathrm{d}x} \, r^2 = 2r \label{eqn:example}
\end{equation}
This can also be used to have equations containing graphical
content. To align the equation number with the middle of the graphic,
rather than the bottom, a minipage may be used.
\begin{equation}
  \begin{minipage}[c]{0.80\linewidth}
    \centering
    As illustrated here, the width of \\
    the minipage needs to allow some  \\
    space for the number to fit in to.
    %\includegraphics{graphic}
  \end{minipage}
  \label{eqn:graphic}
\end{equation}

\section{Experimental}

The usual experimental details should appear here.  This could
include a table, which can be referenced as Table~\ref{tbl:example}.
Notice that the caption is positioned at the top of the table.
\begin{table}
  \caption{An example table}
  \label{tbl:example}
  \begin{tabular}{ll}
    \hline
    Header one  & Header two  \\
    \hline
    Entry one   & Entry two   \\
    Entry three & Entry four  \\
    Entry five  & Entry five  \\
    Entry seven & Entry eight \\
    \hline
  \end{tabular}
\end{table}

Adding notes to tables can be complicated.  Perhaps the easiest
method is to generate these using the basic
\texttt{\textbackslash textsuperscript} and
\texttt{\textbackslash emph} macros, as illustrated (Table~\ref{tbl:notes}).
\begin{table}
  \caption{A table with notes}
  \label{tbl:notes}
  \begin{tabular}{ll}
    \hline
    Header one                            & Header two \\
    \hline
    Entry one\textsuperscript{\emph{a}}   & Entry two  \\
    Entry three\textsuperscript{\emph{b}} & Entry four \\
    \hline
  \end{tabular}

  \textsuperscript{\emph{a}} Some text;
  \textsuperscript{\emph{b}} Some more text.
\end{table}

The example file also loads the optional \textsf{mhchem} package, so
that formulas are easy to input: \texttt{\textbackslash ce\{H2SO4\}}
gives \ce{H2SO4}.  See the use in the bibliography file (when using
titles in the references section).

The use of new commands should be limited to simple things which will
not interfere with the production process.  For example,
\texttt{\textbackslash mycommand} has been defined in this example,
to give italic, mono-spaced text: \mycommand{some text}.

\section{Extra information when writing JACS Communications}

When producing communications for \emph{J.~Am.\ Chem.\ Soc.}, the
class will automatically lay the text out in the style of the
journal. This gives a guide to the length of text that can be
accommodated in such a publication. There are some points to bear in
mind when preparing a JACS Communication in this way.  The layout
produced here is a \emph{model} for the published result, and the
outcome should be taken as a \emph{guide} to the final length. The
spacing and sizing of graphical content is an area where there is
some flexibility in the process.  You should not worry about the
space before and after graphics, which is set to give a guide to the
published size. This is very dependant on the final published layout.

You should be able to use the same source to produce a JACS
Communication and a normal article.  For example, this demonstration
file will work with both \texttt{type=article} and
\texttt{type=communication}. Sections and any abstract are
automatically ignored, although you will get warnings to this effect.

%%%%%%%%%%%%%%%%%%%%%%%%%%%%%%%%%%%%%%%%%%%%%%%%%%%%%%%%%%%%%%%%%%%%%
%% The "Acknowledgement" section can be given in all manuscript
%% classes.  This should be given within the "acknowledgement"
%% environment, which will make the correct section or running title.
%%%%%%%%%%%%%%%%%%%%%%%%%%%%%%%%%%%%%%%%%%%%%%%%%%%%%%%%%%%%%%%%%%%%%
\begin{acknowledgement}

Please use ``The authors thank \ldots'' rather than ``The
authors would like to thank \ldots''.

The author thanks Mats Dahlgren for version one of \textsf{achemso},
and Donald Arseneau for the code taken from \textsf{cite} to move
citations after punctuation. Many users have provided feedback on the
class, which is reflected in all of the different demonstrations
shown in this document.

\end{acknowledgement}

%%%%%%%%%%%%%%%%%%%%%%%%%%%%%%%%%%%%%%%%%%%%%%%%%%%%%%%%%%%%%%%%%%%%%
%% The same is true for Supporting Information, which should use the
%% suppinfo environment.
%%%%%%%%%%%%%%%%%%%%%%%%%%%%%%%%%%%%%%%%%%%%%%%%%%%%%%%%%%%%%%%%%%%%%
\begin{suppinfo}

This will usually read something like: ``Experimental procedures and
characterization data for all new compounds. The class will
automatically add a sentence pointing to the information on-line:

\end{suppinfo}

%%%%%%%%%%%%%%%%%%%%%%%%%%%%%%%%%%%%%%%%%%%%%%%%%%%%%%%%%%%%%%%%%%%%%
%% The appropriate \bibliography command should be placed here.
%% Notice that the class file automatically sets \bibliographystyle
%% and also names the section correctly.
%%%%%%%%%%%%%%%%%%%%%%%%%%%%%%%%%%%%%%%%%%%%%%%%%%%%%%%%%%%%%%%%%%%%%
\bibliography{achemso-demo}

\end{document}