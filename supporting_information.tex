%%%%%%%%%%%%%%%%%%%%%%%%%%%%%%%%%%%%%%%%%%%%%%%%%%%%%%%%%%%%%%%%%%%%%
%% This is a (brief) model paper using the achemso class
%% The document class accepts keyval options, which should include
%% the target journal and optionally the manuscript type. 
%%%%%%%%%%%%%%%%%%%%%%%%%%%%%%%%%%%%%%%%%%%%%%%%%%%%%%%%%%%%%%%%%%%%%
\documentclass[journal=jacsat,manuscript=article]{achemso}

%%%%%%%%%%%%%%%%%%%%%%%%%%%%%%%%%%%%%%%%%%%%%%%%%%%%%%%%%%%%%%%%%%%%%
%% Place any additional packages needed here.  Only include packages
%% which are essential, to avoid problems later. Do NOT use any
%% packages which require e-TeX (for example etoolbox): the e-TeX
%% extensions are not currently available on the ACS conversion
%% servers.
%%%%%%%%%%%%%%%%%%%%%%%%%%%%%%%%%%%%%%%%%%%%%%%%%%%%%%%%%%%%%%%%%%%%%
\usepackage[version=3]{mhchem} % Formula subscripts using \ce{}
\usepackage{siunitx}
\usepackage{graphicx}
\usepackage{xr} % For cross-referencing equations from main.tex
\externaldocument{main}

%%%%%%%%%%%%%%%%%%%%%%%%%%%%%%%%%%%%%%%%%%%%%%%%%%%%%%%%%%%%%%%%%%%%%
%% If issues arise when submitting your manuscript, you may want to
%% un-comment the next line.  This provides information on the
%% version of every file you have used.
%%%%%%%%%%%%%%%%%%%%%%%%%%%%%%%%%%%%%%%%%%%%%%%%%%%%%%%%%%%%%%%%%%%%%
%%\listfiles

%%%%%%%%%%%%%%%%%%%%%%%%%%%%%%%%%%%%%%%%%%%%%%%%%%%%%%%%%%%%%%%%%%%%%
%% Place any additional macros here.  Please use \newcommand* where
%% possible, and avoid layout-changing macros (which are not used
%% when typesetting).
%%%%%%%%%%%%%%%%%%%%%%%%%%%%%%%%%%%%%%%%%%%%%%%%%%%%%%%%%%%%%%%%%%%%%
\newcommand*\mycommand[1]{\texttt{\emph{#1}}}
\setcounter{figure}{0}
\renewcommand{\thefigure}{S\arabic{figure}}
\setcounter{table}{0}
\renewcommand{\thetable}{S\arabic{table}}

%%%%%%%%%%%%%%%%%%%%%%%%%%%%%%%%%%%%%%%%%%%%%%%%%%%%%%%%%%%%%%%%%%%%%
%% Meta-data block
%% ---------------
%% Each author should be given as a separate \author command.
%%
%% Corresponding authors should have an e-mail given after the author
%% name as an \email command. Phone and fax numbers can be given
%% using \phone and \fax, respectively; this information is optional.
%%
%% The affiliation of authors is given after the authors; each
%% \affiliation command applies to all preceding authors not already
%% assigned an affiliation.
%%
%% The affiliation takes an option argument for the short name.  This
%% will typically be something like "University of Somewhere".
%%
%% The \altaffiliation macro should be used for new address, etc.
%% On the other hand, \alsoaffiliation is used on a per author basis
%% when authors are associated with multiple institutions.
%%%%%%%%%%%%%%%%%%%%%%%%%%%%%%%%%%%%%%%%%%%%%%%%%%%%%%%%%%%%%%%%%%%%%
\author{Haruya Ishida}
\affiliation[Kyushu University]
{Department of Aeronautics and Astronautics, Kyushu University, Nishi-Ku, Motooka 744, Fukuoka 819-0395, Japan}
\author{Hideaki Teshima}
\affiliation[Kyushu University]
{Department of Aeronautics and Astronautics, Kyushu University, Nishi-Ku, Motooka 744, Fukuoka 819-0395, Japan}
\email{hteshima05@aero.kyushu-u.ac.jp}
\alsoaffiliation[I2CNER]
{International Institute for Carbon-Neutral Energy Research (WPI-I2CNER), Kyushu University, Nishi-Ku, Motooka 744, Fukuoka 819-0395, Japan}
\author{Koji Takahashi}
\affiliation[Kyushu University]
{Department of Aeronautics and Astronautics, Kyushu University, Nishi-Ku, Motooka 744, Fukuoka 819-0395, Japan}
\alsoaffiliation[I2CNER]
{International Institute for Carbon-Neutral Energy Research (WPI-I2CNER), Kyushu University, Nishi-Ku, Motooka 744, Fukuoka 819-0395, Japan}
\author{Vishwanath Ganesan}
\affiliation[UIUC MechSE]
{Department of Mechanical Science and Engineering, University of Illinois at Urbana-Champaign, Urbana, IL, USA}
\author{Nenad Miljkovic}
\affiliation[I2CNER]
{International Institute for Carbon-Neutral Energy Research (WPI-I2CNER), Kyushu University, Nishi-Ku, Motooka 744, Fukuoka 819-0395, Japan}
\alsoaffiliation[UIUC MechSE]
{Department of Mechanical Science and Engineering, University of Illinois at Urbana-Champaign, Urbana, IL, USA}
\alsoaffiliation[UIUC MRL]
{Materials Research Laboratory, University of Illinois at Urbana-Champaign, Urbana, IL, USA}
\alsoaffiliation[UIUC ECE]
{Department of Electrical and Computer Engineering, University of Illinois at Urbana-Champaign, Urbana, IL, USA}
\alsoaffiliation[iSEE]
{Institute for Sustainability, Energy and Environment (iSEE), University of Illinois at Urbana-Champaign, Urbana, IL, USA}
\alsoaffiliation[ARC]
{Air Conditioning and Refrigeration Center, University of Illinois at Urbana-Champaign, Urbana, IL, USA}

%%%%%%%%%%%%%%%%%%%%%%%%%%%%%%%%%%%%%%%%%%%%%%%%%%%%%%%%%%%%%%%%%%%%%
%% The document title should be given as usual. Some journals require
%% a running title from the author: this should be supplied as an
%% optional argument to \title.
%%%%%%%%%%%%%%%%%%%%%%%%%%%%%%%%%%%%%%%%%%%%%%%%%%%%%%%%%%%%%%%%%%%%%
\title[Title_supporting_information]
  {Supporting Information\\
  Hydrophobic Surfaces Are Not Slippery: Insights from Nanoscale Slip Length Mapping}

%%%%%%%%%%%%%%%%%%%%%%%%%%%%%%%%%%%%%%%%%%%%%%%%%%%%%%%%%%%%%%%%%%%%%
%% Some journals require a list of abbreviations or keywords to be
%% supplied. These should be set up here, and will be printed after
%% the title and author information, if needed.
%%%%%%%%%%%%%%%%%%%%%%%%%%%%%%%%%%%%%%%%%%%%%%%%%%%%%%%%%%%%%%%%%%%%%
\abbreviations{IR,NMR,UV}
\keywords{American Chemical Society, \LaTeX}

%%%%%%%%%%%%%%%%%%%%%%%%%%%%%%%%%%%%%%%%%%%%%%%%%%%%%%%%%%%%%%%%%%%%%
%% The manuscript does not need to include \maketitle, which is
%% executed automatically.
%%%%%%%%%%%%%%%%%%%%%%%%%%%%%%%%%%%%%%%%%%%%%%%%%%%%%%%%%%%%%%%%%%%%%
\begin{document}

%%%%%%%%%%%%%%%%%%%%%%%%%%%%%%%%%%%%%%%%%%%%%%%%%%%%%%%%%%%%%%%%%%%%%
%% Start the main part of the manuscript here.
%%%%%%%%%%%%%%%%%%%%%%%%%%%%%%%%%%%%%%%%%%%%%%%%%%%%%%%%%%%%%%%%%%%%%
\subsection{Relation between Amplitude Decay and Damping Coefficient}

The total damping coefficient, \(\gamma_{\text{total}}\), acting on the probe can be expressed as the sum of the damping due to the probe sphere, \(\gamma_{\text{tip}}\), and the background damping, \(\gamma_{\text{bulk}}\):

\begin{equation}
  \gamma_{\text{total}} = \gamma_{\text{tip}}(h, b_t, b_s) + \gamma_{\text{bulk}}.
  \tag{S1}
  \label{eqn:s1}
\end{equation}

where \(h\) is probe-substrate distance, and \(b_t\) and \(b_s\) are the slip lengths at the probe and substrate surfaces, respectively. The background damping \(\gamma_{\text{bulk}}\) originates from various sources, such as viscous drag on the cantilever beam and friction due to thermal fluctuations \cite{Gauthier1999-su}. However, it remains nearly constant and is therefore treated as a constant here. Using the definition of the quality factor \(Q(h) = k/(\omega\gamma_{\text{total}}(h))\), Eq.~\ref{eqn:s1} can be rewritten as:

\begin{equation}
  \frac{k}{\omega Q(h)} = \gamma_{\text{tip}}(h, b_t, b_s) + \gamma_{\text{bulk}}.
  \tag{S2}
  \label{eqn:s2}
\end{equation}

At sufficiently large \(h \gg R\), the influence of the substrate becomes negligible and the hydrodynamic drag on the probe reduces to the Stokes resistance, \(\gamma_{\text{tip}} = 6\pi\eta R\). Because this drag on a \SI{300}{\nano\metre} spherical tip is orders of magnitude smaller than the viscous damping acting on the micrometer-scale cantilever, \(\gamma_{\text{tip}}\) becomes negligibly small compared to \(\gamma_{\text{bulk}}\), leading to:

\begin{equation}
  \gamma_{\text{total}} = \gamma_{\text{bulk}} = \frac{k}{\omega Q_{\text{bulk}}}.
  \tag{S3}
  \label{eqn:s3}
\end{equation}

Combining Eqs.~\ref{eqn:s2} and \ref{eqn:s3} yields:

\begin{equation}
  \frac{Q_{\text{bulk}}}{Q(h)} = \frac{\gamma_{\text{tip}}(h, b_t, b_s)}{\gamma_{\text{bulk}}} + 1.
  \tag{S4}
  \label{eqn:s4}
\end{equation}

In normal FM-AFM, the oscillation amplitude is stabilized by automatic gain control (AGC). In contrast, in this method, AGC was disabled and the cantilever was driven at a constant power. As a result, the amplitude decay directly reflects the damping, and the following relation between the quality factor and the amplitude holds:

\begin{equation}
  \frac{Q_{\text{bulk}}}{Q(h)} = \frac{A_{\text{bulk}}}{A(h)}.
  \tag{S5}
  \label{eqn:s5}
\end{equation}

Therefore, from Eqs.~\ref{eqn:s4} and \ref{eqn:s5}, the following relation connecting the ratio of amplitude attenuation to the damping coefficient is obtained:

\begin{equation}
  \gamma_{\text{tip}}(h, b_t, b_s) = \gamma_{\text{bulk}}\left(\frac{A_{\text{bulk}}}{A(h)} - 1\right).
  \tag{S6}
  \label{eqn:s6}
\end{equation}

\subsection{Comparison of Sensitivity between Conventional and Proposed Methods}

We compared the sensitivity of slip length measurements between conventional contact-mode AFM and our newly proposed FM-AFM. With contact-mode AFM, the cantilever deflection \(x\) is measured as a sensor voltage, and the relationship between the deflection sensor voltage \(V_x\) and \(x\) is expressed using the sensitivity \(s\) as \(x = sV_x\). Furthermore, to determine the viscous drag in Eq.~\ref{eqn:hydrodynamic force} from the restoring force of the cantilever, \(kx\) (\(k\) is the spring constant), the following relation holds:

\begin{equation}
  F = -ksV_x = -\frac{6\pi\eta R^2 \dot{h}}{h} f^{*}(b_t, b_s).
  \tag{S7}
  \label{eqn:s7}
\end{equation}

In contrast, in the FM-AFM method, the oscillation amplitude \(A\) is recorded as a sensor voltage \(V_A\), with the relationship \(A = sV_A/\sqrt{2}\). From Eqs.~\ref{eqn:damping coefficient amplitude} and \ref{eqn:damping coefficient slip} of the manuscript, the following relation is obtained:

\begin{equation}
  \frac{6\pi\eta R^2}{h} f^{*}(b_t, b_s) = \gamma_{\text{bulk}}\left(\frac{V_{A,\text{bulk}}}{V_A(h)} - 1\right).
  \tag{S8}
  \label{eqn:s8}
\end{equation}

Here, the measurement sensitivity is defined as the change in signal voltage with respect to the slip length, \(\partial V/\partial b_s\). Differentiating Eqs.~\ref{eqn:s7} and \ref{eqn:s8} yields the sensitivities for each method, \(S_{\text{contact}}\) and \(S_{\text{FM}}\):

\begin{equation}
  S_{\text{contact}} = \frac{\partial V_x}{\partial b_s} = -\frac{6\pi\eta R^2 \dot{h}}{k s h}\frac{\partial f^{*}}{\partial b_s}.
  \tag{S9}
  \label{eqn:s9}
\end{equation}

\begin{equation}
  S_{\text{FM}} = \frac{\partial V_A}{\partial b_s} = -\frac{6\pi\eta R^2 V_{A,\text{bulk}}}{h\gamma_{\text{bulk}}}\left(1 + \frac{6\pi\eta R^2 f^{*}}{h\gamma_{\text{bulk}}}\right)^{-2}\frac{\partial f^{*}}{\partial b_s}.
  \tag{S10}
  \label{eqn:s10}
\end{equation}

From Eqs.~\ref{eqn:s9} and \ref{eqn:s10}, the ratio of sensitivities is given by:

\begin{equation}
  \frac{S_{\text{FM}}}{S_{\text{contact}}} = \frac{V_{A,\text{bulk}} k s}{\gamma_{\text{bulk}} \dot{h}}\left(1 + \frac{6\pi\eta R^2 f^{*}}{h\gamma_{\text{bulk}}}\right)^{-2}.
  \tag{S11}
  \label{eqn:s11}
\end{equation}

The sensitivity ratio was calculated to be \(S_{\text{FM}}/S_{\text{contact}} = 159\) using typical physical parameters in the measurements: \(V_{A,\text{bulk}} =\) \SI{0.2}{\volt}, \(k =\) \SI{20}{\newton\per\metre}, \(s =\) \SI{10}{\nano\metre\per\volt}, \(\gamma_{\text{bulk}} =\) \SI{2.3}{\micro\newton\second\per\metre}, \(\dot{h} =\) \SI{100}{\micro\metre\per\second}, \(R =\) \SI{300}{\nano\metre}, \(\eta =\) \SI{1}{\milli\pascal\second}, \(f^{*} = 0.63\), and \(h =\) \SI{10}{\nano\metre}. The \(f^{*}\) value corresponds to the slip length of \(b_t =\) \SI{0}{\nano\metre} and \(b_s =\) \SI{10}{\nano\metre}, and the tip substrate distance \(h =\) \SI{10}{\nano\metre} is chosen as a representative value among the range of distances used in the calculation. This indicates that the present method is 159 times more sensitive to slip length than the conventional approach. Consequently, slip length can be measured using much smaller colloidal probes than before.

\subsection{Materials}

\subsubsection{Fabrication of Hydrophilic-Hydrophobic Composite Substrates}

A thermally grown silica surface on a Si wafer was spin-coated with an electron-beam resist (ZEP520A, ZEON Corporation, Japan) at \SI{5000}{\per\minute} to form a \SI{300}{\nano\metre}-thick film. Using an electron-beam lithography system (JSM-6360, JEOL Ltd., Japan / BEAM DRAW, Tokyo Technology Inc., Japan), square patterns of \SI{1}{\micro\metre}\(\times\)\SI{1}{\micro\metre} were drawn at a \SI{2}{\micro\metre} pitch, and the resist in the exposed areas was removed by development. Subsequently, a \SI{5}{\nano\metre}-thick indium tin oxide (ITO) film was deposited by magnetron sputtering (MS-4B-3T, Cosmo System Co., Ltd., Japan). The electron-beam resist was then lifted off by rinsing in dimethylacetamide for \SI{5}{\minute}, leaving ITO patches on the silica substrate. The patterned ITO substrate was treated with \(\mathrm{O}_2\) plasma (PR500, Yamato Scientific Co., Japan) and then immersed in a \SI{1}{\milli\mole\per\litre} ethanol solution of 1H,1H,2H,2H-perfluorooctanephosphonic acid (FOPA) for \SI{1}{\hour} to form a self-assembled monolayer (SAM). Since FOPA SAMs are easily hydrolyzed on silica but remain stable on ITO surfaces, maintaining hydrophobicity for hours in water \cite{Thissen2012-xt, Jo2011-nw}, the fabricated composite substrate was immersed in water for \SI{3}{\hour} to selectively remove FOPA from the silica surface.

\subsubsection{Fabrication of Teflon Substrates}

Silica substrates cut to \SI{1}{\centi\metre}\(\times\)\SI{1}{\centi\metre} were ultrasonically cleaned in acetone (US-1KS, SND Co., Ltd., Japan) and treated with \(\mathrm{O}_2\) plasma. After baking at \SI{120}{\degreeCelsius} for \SI{5}{\minute} to remove surface moisture, the substrates were immersed for \SI{1}{\hour} in a \SI{0.05}{\percent} (w/w) toluene solution of FDTS to form an FDTS monolayer. On top of the FDTS-treated substrate, a \SI{0.6}{\percent} (w/w) solution of Teflon AF1600X (DuPont Inc., Wilmington, DE, USA) in FC-770 (3M, St. Paul, MN, USA) was dropped, spin-coated at \SI{4000}{\per\minute} for \SI{1}{\minute}, and baked at \SI{175}{\degreeCelsius} for \SI{10}{\minute}. The FDTS underlayer was used to improve the adhesion of the Teflon coating \cite{Makohliso1998-av}.

\begin{figure}[p]
  \centering
  \includegraphics[width=1.0\linewidth]{figs/mapping_colormaps.png}
  \caption{ (a-d) Topography and (e-h) corresponding slip-length maps of (a, e) mica, (b, f) FDTS, (c, g) Teflon, and (d, h) HOPG. The \(R_a\) values were \SI{0.27}{\nano\metre}, \SI{0.29}{\nano\metre}, \SI{0.64}{\nano\metre}, and \SI{0.35}{\nano\metre} for each surface, respectively. Only the result for HOPG (d, h) was obtained in a \SI{10}{\milli\mole\per\litre} KCl solution and the others were in DI water.}
  \label{figS:mapping_colormaps}
\end{figure}

\begin{table}[p]
  \caption{Contact angle and slip length for various substrates}
  \label{tbl:contact_angle_slip}
  \begin{tabular}{lcc}
    \hline
    Substrate & Contact Angle \(\theta\) (\si{\degree}) & Slip length (\si{\nano\metre}) \\
    \hline
    Mica & \(< \SI{3}{\degree}\) & \(-1.7 \pm \SI{0.3}{\nano\metre}\) \\
    Oxygen plasma-treated silica & \SI{28.1}{\degree} & \(-0.7 \pm \SI{0.6}{\nano\metre}\) \\
    FDTS & \SI{97.9}{\degree} & \(-1.7 \pm \SI{0.2}{\nano\metre}\) \\
    FOPA & \SI{105.0}{\degree} & \(-0.9 \pm \SI{0.6}{\nano\metre}\) \\
    Teflon & \SI{111.1}{\degree} & \(0.8 \pm \SI{0.8}{\nano\metre}\) \\
    HOPG (DI water) & \SI{59.0}{\degree} & \(43.2 \pm \SI{5.8}{\nano\metre}\) \\
    HOPG (\SI{10}{\milli\mole\per\litre} KCl aq.) & \SI{62.8}{\degree} & \(-0.1 \pm \SI{0.7}{\nano\metre}\) \\
    HOPG (\SI{10}{\milli\mole\per\litre} NaCl aq.) & \SI{57.0}{\degree} & \(-2.3 \pm \SI{0.6}{\nano\metre}\) \\
    \hline
  \end{tabular}
\end{table}

%%%%%%%%%%%%%%%%%%%%%%%%%%%%%%%%%%%%%%%%%%%%%%%%%%%%%%%%%%%%%%%%%%%%%
%% The appropriate \bibliography command should be placed here.
%% Notice that the class file automatically sets \bibliographystyle
%% and also names the section correctly.
%%%%%%%%%%%%%%%%%%%%%%%%%%%%%%%%%%%%%%%%%%%%%%%%%%%%%%%%%%%%%%%%%%%%%
\clearpage
\bibliography{main}

\end{document}